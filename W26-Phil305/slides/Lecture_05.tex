% Options for packages loaded elsewhere
% Options for packages loaded elsewhere
\PassOptionsToPackage{unicode}{hyperref}
\PassOptionsToPackage{hyphens}{url}
%
\documentclass[
  ignorenonframetext,
  handout]{beamer}
\newif\ifbibliography
\usepackage{pgfpages}
\setbeamertemplate{caption}[numbered]
\setbeamertemplate{caption label separator}{: }
\setbeamercolor{caption name}{fg=normal text.fg}
\beamertemplatenavigationsymbolsempty
% remove section numbering
\setbeamertemplate{part page}{
  \centering
  \begin{beamercolorbox}[sep=16pt,center]{part title}
    \usebeamerfont{part title}\insertpart\par
  \end{beamercolorbox}
}
\setbeamertemplate{section page}{
  \centering
  \begin{beamercolorbox}[sep=12pt,center]{section title}
    \usebeamerfont{section title}\insertsection\par
  \end{beamercolorbox}
}
\setbeamertemplate{subsection page}{
  \centering
  \begin{beamercolorbox}[sep=8pt,center]{subsection title}
    \usebeamerfont{subsection title}\insertsubsection\par
  \end{beamercolorbox}
}
% Prevent slide breaks in the middle of a paragraph
\widowpenalties 1 10000
\raggedbottom
\usepackage{iftex}
\ifPDFTeX
  \usepackage[T1]{fontenc}
  \usepackage[utf8]{inputenc}
  \usepackage{textcomp} % provide euro and other symbols
\else % if luatex or xetex
  \usepackage{unicode-math} % this also loads fontspec
  \defaultfontfeatures{Scale=MatchLowercase}
  \defaultfontfeatures[\rmfamily]{Ligatures=TeX,Scale=1}
\fi
\usepackage{lmodern}

\usetheme[]{metropolis}
\usecolortheme[]{default}
\usefonttheme[]{structurebold}
\ifPDFTeX\else
  % xetex/luatex font selection
\fi
% Use upquote if available, for straight quotes in verbatim environments
\IfFileExists{upquote.sty}{\usepackage{upquote}}{}
\IfFileExists{microtype.sty}{% use microtype if available
  \usepackage[]{microtype}
  \UseMicrotypeSet[protrusion]{basicmath} % disable protrusion for tt fonts
}{}
\makeatletter
\@ifundefined{KOMAClassName}{% if non-KOMA class
  \IfFileExists{parskip.sty}{%
    \usepackage{parskip}
  }{% else
    \setlength{\parindent}{0pt}
    \setlength{\parskip}{6pt plus 2pt minus 1pt}}
}{% if KOMA class
  \KOMAoptions{parskip=half}}
\makeatother


\usepackage{longtable,booktabs,array}
\usepackage{calc} % for calculating minipage widths
\usepackage{caption}
% Make caption package work with longtable
\makeatletter
\def\fnum@table{\tablename~\thetable}
\makeatother
\usepackage{graphicx}
\makeatletter
\newsavebox\pandoc@box
\newcommand*\pandocbounded[1]{% scales image to fit in text height/width
  \sbox\pandoc@box{#1}%
  \Gscale@div\@tempa{\textheight}{\dimexpr\ht\pandoc@box+\dp\pandoc@box\relax}%
  \Gscale@div\@tempb{\linewidth}{\wd\pandoc@box}%
  \ifdim\@tempb\p@<\@tempa\p@\let\@tempa\@tempb\fi% select the smaller of both
  \ifdim\@tempa\p@<\p@\scalebox{\@tempa}{\usebox\pandoc@box}%
  \else\usebox{\pandoc@box}%
  \fi%
}
% Set default figure placement to htbp
\def\fps@figure{htbp}
\makeatother





\setlength{\emergencystretch}{3em} % prevent overfull lines

\providecommand{\tightlist}{%
  \setlength{\itemsep}{0pt}\setlength{\parskip}{0pt}}



 


% Minimal header for tableaux only

% Tableaux system
\RequirePackage[tableaux]{prooftrees}
\forestset{line numbering, close with = x}

% Open Logic commands for tableaux
\usepackage{open-logic-minimal}

% Define the tableau environment
\newenvironment{oltableau}{\center\tableau{}}{\endtableau\endcenter}

% Define True/False symbols
\usepackage[bb=boondox]{mathalfa}
\DeclareMathAlphabet{\mathbx}{U}{BOONDOX-ds}{m}{n}
\def\True{\mathbb{T}}
\def\False{\mathbb{F}}

% Spacing around tableaux
\usepackage{etoolbox}
\AfterEndEnvironment{oltableau}{\vspace{9pt}}
\BeforeBeginEnvironment{oltableau}{\vspace{9pt}}

% Color definitions (if you need them later)
\definecolor{darkblue}{rgb}{0,0,0.8}
\definecolor{dodgerblue}{RGB}{30,144,255}
\usepackage{fontspec}
\setmainfont{Arial}
\setmonofont[Scale=0.8]{Fira Mono}
\definecolor{primary}{RGB}{0, 102, 153}
\definecolor{secondary}{RGB}{102, 153, 204}
\setbeamercolor{frametitle}{fg=white}
\setbeamercolor{title}{fg=primary}
\setbeamertemplate{footline}[frame number]
\setbeamertemplate{navigation symbols}{}"
\makeatletter
\@ifpackageloaded{caption}{}{\usepackage{caption}}
\AtBeginDocument{%
\ifdefined\contentsname
  \renewcommand*\contentsname{Table of contents}
\else
  \newcommand\contentsname{Table of contents}
\fi
\ifdefined\listfigurename
  \renewcommand*\listfigurename{List of Figures}
\else
  \newcommand\listfigurename{List of Figures}
\fi
\ifdefined\listtablename
  \renewcommand*\listtablename{List of Tables}
\else
  \newcommand\listtablename{List of Tables}
\fi
\ifdefined\figurename
  \renewcommand*\figurename{Figure}
\else
  \newcommand\figurename{Figure}
\fi
\ifdefined\tablename
  \renewcommand*\tablename{Table}
\else
  \newcommand\tablename{Table}
\fi
}
\@ifpackageloaded{float}{}{\usepackage{float}}
\floatstyle{ruled}
\@ifundefined{c@chapter}{\newfloat{codelisting}{h}{lop}}{\newfloat{codelisting}{h}{lop}[chapter]}
\floatname{codelisting}{Listing}
\newcommand*\listoflistings{\listof{codelisting}{List of Listings}}
\makeatother
\makeatletter
\makeatother
\makeatletter
\@ifpackageloaded{caption}{}{\usepackage{caption}}
\@ifpackageloaded{subcaption}{}{\usepackage{subcaption}}
\makeatother

\usepackage{bookmark}
\IfFileExists{xurl.sty}{\usepackage{xurl}}{} % add URL line breaks if available
\urlstyle{same}
\hypersetup{
  pdftitle={Truth Trees},
  hidelinks,
  pdfcreator={LaTeX via pandoc}}


\title{Truth Trees}
\author{}
\date{2026-01-26}

\begin{document}
\frame{\titlepage}


\section{Introducing Trees}\label{introducing-trees}

\begin{frame}{Overview}
\phantomsection\label{overview}
This section introduces a new way of testing for validity - truth trees.

\begin{block}{Associated Reading}
\phantomsection\label{associated-reading}
Boxes and Diamonds, section 2.1
\end{block}
\end{frame}

\begin{frame}{What Tableaux Are}
\phantomsection\label{what-tableaux-are}
\begin{itemize}
\tightlist
\item
  A way for determining whether some combinations are logically
  possible.
\item
  That can be used for determining whether some arguments are valid - if
  the truth of the premises and the falsity of the conclusion is not
  logically possible; then the argument is valid.
\end{itemize}
\end{frame}

\begin{frame}{Structure}
\phantomsection\label{structure}
\begin{itemize}
\tightlist
\item
  Start out writing the things you care about.\pause
\item
  Each time one of those things implies that some other things must be
  the case, write those down too.
\item
  For example, if you write down that \(A \wedge B\) is true, also write
  down that \(A\) is true and that \(B\) is true. \pause
\item
  Each time there are multiple ways to make something you've written
  true, create multiple branches for those ways.
\item
  For example, if you write down that \(A \vee B\) is true, create a
  branch where \(A\) is true, and a branch where \(B\) is true.
\end{itemize}
\end{frame}

\begin{frame}{Closing}
\phantomsection\label{closing}
\begin{itemize}[<+->]
\tightlist
\item
  A branch of a tableau is closed if it contains that some particular
  claim has incompatible truth values.
\item
  For now, this means that one sentence is both true and false.
\item
  The whole tableau is closed if every branch is closed.
\end{itemize}
\end{frame}

\begin{frame}{What Closure Means}
\phantomsection\label{what-closure-means}
\begin{itemize}
\tightlist
\item
  If the tableau is closed, then the initial assumptions cannot be true
  together.
\item
  If you are evaluating an argument, this means that the argument is
  valid.
\end{itemize}

\pause

If you start the tableau by just saying that one sentence is false, the
closure of the tableau means that that sentence is a logical truth.
\end{frame}

\begin{frame}{Benefits of Tableaux}
\phantomsection\label{benefits-of-tableaux}
Tableaux have two big benefits over truth tables.

\begin{enumerate}[<+->]
\tightlist
\item
  They don't grow exponentially when you increase the number of
  variables.
\item
  They can be generalised to things beyond propositional logic.
\end{enumerate}

\pause

We are introducing them here because of point 2.
\end{frame}

\begin{frame}{Open Tableaux}
\phantomsection\label{open-tableaux}
Here is something the book doesn't make a big deal of, but is kind of
important.

\begin{itemize}
\tightlist
\item
  A closed tableau can show that an argument is valid.
\item
  An open \textbf{and completed} tableau can show that an argument is
  invalid.
\end{itemize}
\end{frame}

\begin{frame}{Open Tableaux}
\phantomsection\label{open-tableaux-1}
\begin{itemize}
\tightlist
\item
  The trick here is that it's hard to tell when a tableau is completed
  in the relevant sense.
\item
  This will be easier to illustrate in practice than in theory, so let's
  start building tableau up.
\end{itemize}
\end{frame}

\begin{frame}{Signs}
\phantomsection\label{signs}
The system we are using is what is called a \textbf{signed tableau}
system.

\begin{itemize}
\tightlist
\item
  That means that every line consists of two parts.
\item
  The bigger, second, part is a sentence.
\item
  The first part is a \textbf{sign}, which for now is a truth value.
\item
  That is, it is either \(\mathbb{T}\) or \(\mathbb{F}\).
\end{itemize}
\end{frame}

\begin{frame}{What Lines Mean}
\phantomsection\label{what-lines-mean}
So each line either says that a particular sentence is true, or says
that it is false.

\begin{itemize}
\tightlist
\item
  The book for some reason includes the word `might' here.
\item
  That's misleading; what they should say is that each line says what is
  true given (a) the starting assumptions and (b) the assumptions we
  made for branching purposes.
\end{itemize}
\end{frame}

\begin{frame}{Numbering And Annotation}
\phantomsection\label{numbering-and-annotation}
The system we're using also has two other distinctive features.

\begin{enumerate}
\tightlist
\item
  We number each of the lines.
\item
  We list why we're writing down each line to the right of the tree.
\end{enumerate}

Both of these are somewhat idiosyncratic, though not abnormal. Unlike
the truth tables, there just aren't well defined conventions for how to
write these things out.
\end{frame}

\section{Rules}\label{rules}

\begin{frame}{Overview}
\phantomsection\label{overview-1}
This section introduces the rules we use for building up truth trees.

\begin{block}{Associated Reading}
\phantomsection\label{associated-reading-1}
Boxes and Diamonds, sections 2.2-2.3.
\end{block}
\end{frame}

\begin{frame}{What Rules Do}
\phantomsection\label{what-rules-do}
The rules tell you what new lines to write down given the lines you've
already got.

\begin{itemize}
\tightlist
\item
  To some extent they simply have to be memorised.
\item
  But hopefully they are all (except for the rules about
  \(\rightarrow\)) fairly intuitive.
\end{itemize}
\end{frame}

\begin{frame}{Rules for \(\neg\)}
\phantomsection\label{rules-for-neg}
\begin{columns}[T]
\begin{column}{0.5\linewidth}
\begin{oltableau}
[\sFmla{\True}{\neg A}, 
    [\sFmla{\False}{A}, just = {\TRule{\True}{\neg}[1]}]
]
\end{oltableau}
\end{column}

\begin{column}{0.5\linewidth}
\begin{oltableau}
[\sFmla{\False}{\neg A}, 
    [\sFmla{\True}{A}, just = {\TRule{\False}{\neg}[1]}]
]
\end{oltableau}
\end{column}
\end{columns}
\end{frame}

\begin{frame}{Rules for \(\neg\)}
\phantomsection\label{rules-for-neg-1}
\begin{itemize}
\tightlist
\item
  Note that the line numbers are just for illustration, and are
  arbitrary in two senses.
\item
  First, you apply the rule wherever a sentence like \(\neg A\) appears,
  not just at line 1.
\item
  Second, you don't need to apply the rules immediately, so the
  successor line could come later than 2.
\end{itemize}
\end{frame}

\begin{frame}{Rule for true \(\wedge\) sentence}
\phantomsection\label{rule-for-true-wedge-sentence}
\begin{oltableau}
[\sFmla{\True}{A \wedge B}, 
    [\sFmla{\True}{A}, just = {\TRule{\True}{\wedge}[1]}
      [\sFmla{\True}{B}, just = {\TRule{\True}{\wedge}[1]}]
    ]
]
\end{oltableau}
\end{frame}

\begin{frame}{Rule for true \(\wedge\) sentence}
\phantomsection\label{rule-for-true-wedge-sentence-1}
When you have a true \(\wedge\) sentence, you can write down that the
sentences either side of it are true.
\end{frame}

\begin{frame}{Rule for true \(\vee\) sentence}
\phantomsection\label{rule-for-true-vee-sentence}
\begin{oltableau}
[\sFmla{\True}{A \vee B}, 
    [\sFmla{\True}{A}, just = {\TRule{\True}{\vee}[1]}]
    [\sFmla{\True}{B}, just = {\TRule{\True}{\vee}[1]}]
]
\end{oltableau}
\end{frame}

\begin{frame}{Rule for true \(\vee\) sentence}
\phantomsection\label{rule-for-true-vee-sentence-1}
\begin{itemize}
\tightlist
\item
  When you have a true \(\vee\) sentence, you create two
  \textbf{branches}.
\item
  The way to read the tree is that at least one of the branches must be
  all true. \pause
\item
  The `trunk' above the branching (in this case just line 1), is part of
  both branches.
\item
  Branches are inclusive; you are saying that at least one branch is
  true, not that precisely one is.
\end{itemize}
\end{frame}

\begin{frame}{Rule for false \(\wedge\) sentence}
\phantomsection\label{rule-for-false-wedge-sentence}
\begin{oltableau}
[\sFmla{\False}{A \wedge B}, 
    [\sFmla{\False}{A}, just = {\TRule{\False}{\wedge}[1]}]
    [\sFmla{\False}{B}, just = {\TRule{\False}{\wedge}[1]}]
]
\end{oltableau}
\end{frame}

\begin{frame}{Rule for false \(\wedge\) sentence}
\phantomsection\label{rule-for-false-wedge-sentence-1}
\begin{itemize}
\tightlist
\item
  If an \(\wedge\) sentence is false, then we know that one or other (or
  both) of the sides are false.
\item
  So we create two branches, one where each side is false.
\end{itemize}
\end{frame}

\begin{frame}{Rule for false \(\vee\) sentence}
\phantomsection\label{rule-for-false-vee-sentence}
\begin{oltableau}
[\sFmla{\False}{A \vee B}, 
    [\sFmla{\False}{A}, just = {\TRule{\False}{\vee}[1]}
      [\sFmla{\False}{B}, just = {\TRule{\False}{\vee}[1]}]
    ]
]
\end{oltableau}
\end{frame}

\begin{frame}{Rule for false \(\vee\) sentence}
\phantomsection\label{rule-for-false-vee-sentence-1}
When you have a false \(\vee\) sentence, you know that each side is
false, so you write down that they are both false.
\end{frame}

\begin{frame}{Justifying the rule for false \(\vee\) sentences}
\phantomsection\label{justifying-the-rule-for-false-vee-sentences}
Recall the truth table for \(\vee\)

\begin{longtable}[]{@{}
  >{\raggedright\arraybackslash}p{(\linewidth - 10\tabcolsep) * \real{0.1304}}
  >{\raggedright\arraybackslash}p{(\linewidth - 10\tabcolsep) * \real{0.1304}}
  >{\raggedright\arraybackslash}p{(\linewidth - 10\tabcolsep) * \real{0.1304}}
  >{\raggedright\arraybackslash}p{(\linewidth - 10\tabcolsep) * \real{0.1304}}
  >{\raggedright\arraybackslash}p{(\linewidth - 10\tabcolsep) * \real{0.3478}}
  >{\raggedright\arraybackslash}p{(\linewidth - 10\tabcolsep) * \real{0.1304}}@{}}
\caption{Truth table for \(\vee\)}\label{tbl-or}\tabularnewline
\toprule\noalign{}
\begin{minipage}[b]{\linewidth}\raggedright
A
\end{minipage} & \begin{minipage}[b]{\linewidth}\raggedright
B
\end{minipage} & \begin{minipage}[b]{\linewidth}\raggedright
\end{minipage} & \begin{minipage}[b]{\linewidth}\raggedright
A
\end{minipage} & \begin{minipage}[b]{\linewidth}\raggedright
\(\vee\)
\end{minipage} & \begin{minipage}[b]{\linewidth}\raggedright
B
\end{minipage} \\
\midrule\noalign{}
\endfirsthead
\toprule\noalign{}
\begin{minipage}[b]{\linewidth}\raggedright
A
\end{minipage} & \begin{minipage}[b]{\linewidth}\raggedright
B
\end{minipage} & \begin{minipage}[b]{\linewidth}\raggedright
\end{minipage} & \begin{minipage}[b]{\linewidth}\raggedright
A
\end{minipage} & \begin{minipage}[b]{\linewidth}\raggedright
\(\vee\)
\end{minipage} & \begin{minipage}[b]{\linewidth}\raggedright
B
\end{minipage} \\
\midrule\noalign{}
\endhead
\(\mathbb{T}\) & \(\mathbb{T}\) & & \(\mathbb{T}\) &
\(\color{red}{\mathbb{T}}\) & \(\mathbb{T}\) \\
\(\mathbb{T}\) & \(\mathbb{F}\) & & \(\mathbb{T}\) &
\(\color{red}{\mathbb{T}}\) & \(\mathbb{F}\) \\
\(\mathbb{F}\) & \(\mathbb{T}\) & & \(\mathbb{F}\) &
\(\color{red}{\mathbb{T}}\) & \(\mathbb{T}\) \\
\(\mathbb{F}\) & \(\mathbb{F}\) & & \(\mathbb{F}\) &
\(\color{red}{\mathbb{F}}\) & \(\mathbb{F}\) \\
\bottomrule\noalign{}
\end{longtable}

\begin{itemize}
\tightlist
\item
  The only line where the whole sentence is \(\color{red}{\mathbb{F}}\)
  is line 4.
\item
  So if a \(\vee\) sentence is \(\color{red}{\mathbb{F}}\), we know that
  we're on line 4.
\item
  And on line 4, both \(A\) and \(B\) are false.
\end{itemize}
\end{frame}

\begin{frame}{Rule for false \(\rightarrow\) sentence}
\phantomsection\label{rule-for-false-rightarrow-sentence}
\begin{oltableau}
[\sFmla{\False}{A \rightarrow B}, 
    [\sFmla{\True}{A}, just = {\TRule{\False}{\rightarrow}[1]}
      [\sFmla{\False}{B}, just = {\TRule{\False}{\rightarrow}[1]}]
    ]
]
\end{oltableau}
\end{frame}

\begin{frame}{Rule for false \(\rightarrow\) sentence}
\phantomsection\label{rule-for-false-rightarrow-sentence-1}
When you have a false \(\rightarrow\) sentence, you know that the left
side is true and the right side is false, so you write those things
down.
\end{frame}

\begin{frame}{Justifying the rule for false \(\rightarrow\) sentences}
\phantomsection\label{justifying-the-rule-for-false-rightarrow-sentences}
Recall the truth table for \(\rightarrow\)

\begin{longtable}[]{@{}
  >{\raggedright\arraybackslash}p{(\linewidth - 10\tabcolsep) * \real{0.1304}}
  >{\raggedright\arraybackslash}p{(\linewidth - 10\tabcolsep) * \real{0.1304}}
  >{\raggedright\arraybackslash}p{(\linewidth - 10\tabcolsep) * \real{0.1304}}
  >{\raggedright\arraybackslash}p{(\linewidth - 10\tabcolsep) * \real{0.1304}}
  >{\raggedright\arraybackslash}p{(\linewidth - 10\tabcolsep) * \real{0.3478}}
  >{\raggedright\arraybackslash}p{(\linewidth - 10\tabcolsep) * \real{0.1304}}@{}}
\caption{Truth table for \(\rightarrow\)}\tabularnewline
\toprule\noalign{}
\begin{minipage}[b]{\linewidth}\raggedright
A
\end{minipage} & \begin{minipage}[b]{\linewidth}\raggedright
B
\end{minipage} & \begin{minipage}[b]{\linewidth}\raggedright
\end{minipage} & \begin{minipage}[b]{\linewidth}\raggedright
A
\end{minipage} & \begin{minipage}[b]{\linewidth}\raggedright
\(\rightarrow\)
\end{minipage} & \begin{minipage}[b]{\linewidth}\raggedright
B
\end{minipage} \\
\midrule\noalign{}
\endfirsthead
\toprule\noalign{}
\begin{minipage}[b]{\linewidth}\raggedright
A
\end{minipage} & \begin{minipage}[b]{\linewidth}\raggedright
B
\end{minipage} & \begin{minipage}[b]{\linewidth}\raggedright
\end{minipage} & \begin{minipage}[b]{\linewidth}\raggedright
A
\end{minipage} & \begin{minipage}[b]{\linewidth}\raggedright
\(\rightarrow\)
\end{minipage} & \begin{minipage}[b]{\linewidth}\raggedright
B
\end{minipage} \\
\midrule\noalign{}
\endhead
\(\mathbb{T}\) & \(\mathbb{T}\) & & \(\mathbb{T}\) &
\(\color{red}{\mathbb{T}}\) & \(\mathbb{T}\) \\
\(\mathbb{T}\) & \(\mathbb{F}\) & & \(\mathbb{T}\) &
\(\color{red}{\mathbb{F}}\) & \(\mathbb{F}\) \\
\(\mathbb{F}\) & \(\mathbb{T}\) & & \(\mathbb{F}\) &
\(\color{red}{\mathbb{T}}\) & \(\mathbb{T}\) \\
\(\mathbb{F}\) & \(\mathbb{F}\) & & \(\mathbb{F}\) &
\(\color{red}{\mathbb{T}}\) & \(\mathbb{F}\) \\
\bottomrule\noalign{}
\end{longtable}

\begin{itemize}
\tightlist
\item
  The only line where the whole sentence is {\(\mathbb{F}\)} is line 2.
\item
  So if a \(\rightarrow\) sentence is {\(\mathbb{F}\)}, we know that
  we're on line 2.
\item
  And on line 2,\(A\) is true and \(B\) are false.
\end{itemize}
\end{frame}

\begin{frame}{Rule for true \(\rightarrow\) sentence}
\phantomsection\label{rule-for-true-rightarrow-sentence}
\begin{oltableau}
[\sFmla{\True}{A \rightarrow B}, 
    [\sFmla{\False}{A}, just = {\TRule{\True}{\rightarrow}[1]}]
    [\sFmla{\True}{B}, just = {\TRule{\True}{\rightarrow}[1]}]
]
\end{oltableau}
\end{frame}

\begin{frame}{Rule for true \(\rightarrow\) sentence}
\phantomsection\label{rule-for-true-rightarrow-sentence-1}
\begin{itemize}
\tightlist
\item
  When you have a true \(\rightarrow\) sentence, you create two
  \textbf{branches}.
\item
  On the first, \(A\) is false. That covers lines 3 and 4 of the truth
  table.
\item
  On the second, \(B\) is true. That covers lines 1 and 3 of the truth
  table.
\item
  Between them, they cover lines 1, 3 and 4 of the truth table.
\item
  And those are the lines where \(A \rightarrow B\) is true.
\end{itemize}
\end{frame}

\section{Examples}\label{examples}

\begin{frame}{Overview}
\phantomsection\label{overview-2}
This section works through some examples of using tableaux.
\end{frame}

\begin{frame}{Example 1: Showing a Tautology}
\phantomsection\label{example-1-showing-a-tautology}
Let's show that \((P \rightarrow Q) \vee (Q \rightarrow P)\) is a
tautology.

\begin{itemize}
\tightlist
\item
  You might remember this from last time - we showed it was a tautology
  using a truth table.
\item
  Now we'll use a tableau instead.
\end{itemize}
\end{frame}

\begin{frame}{Strategy}
\phantomsection\label{strategy}
To show a sentence is a tautology using tableaux:

\begin{itemize}[<+->]
\tightlist
\item
  Start by assuming the sentence is \textbf{false}.
\item
  If the tableau closes, then it's impossible for the sentence to be
  false.
\item
  Remember, a tableau closes if every branch has at least one
  inconsistency.
\item
  If the tableau closes, the sentence must be true - it's a tautology.
\end{itemize}
\end{frame}

\begin{frame}{Building the Tree}
\phantomsection\label{building-the-tree}
We start by writing that \((P \rightarrow Q) \vee (Q \rightarrow P)\) is
false.

\begin{oltableau}
[\sFmla{\False}{(P \rightarrow Q) \vee (Q \rightarrow P)}]
\end{oltableau}
\end{frame}

\begin{frame}{Applying the False \(\vee\) Rule}
\phantomsection\label{applying-the-false-vee-rule}
Since we have a false disjunction, we know both disjuncts must be false.

\begin{oltableau}
[\sFmla{\False}{(P \rightarrow Q) \vee (Q \rightarrow P)}, 
  [\sFmla{\False}{P \rightarrow Q}, just = {\TRule{\False}{\vee}[1]}
    [\sFmla{\False}{Q \rightarrow P}, just = {\TRule{\False}{\vee}[1]}]
  ]
]
\end{oltableau}
\end{frame}

\begin{frame}{Applying the False \(\rightarrow\) Rule (First Time)}
\phantomsection\label{applying-the-false-rightarrow-rule-first-time}
When a conditional is false, the antecedent is true and the consequent
is false.

So from \(\mathbb{F}(P \rightarrow Q)\), we get \(\mathbb{T}P\) and
\(\mathbb{F}Q\).
\end{frame}

\begin{frame}{Applying the False \(\rightarrow\) Rule (First Time)}
\phantomsection\label{applying-the-false-rightarrow-rule-first-time-1}
\begin{oltableau}
[\sFmla{\False}{(P \rightarrow Q) \vee (Q \rightarrow P)}, 
  [\sFmla{\False}{P \rightarrow Q}, just = {\TRule{\False}{\vee}[1]}
    [\sFmla{\False}{Q \rightarrow P}, just = {\TRule{\False}{\vee}[1]}
      [\sFmla{\True}{P}, just = {\TRule{\False}{\rightarrow}[2]}
        [\sFmla{\False}{Q}, just = {\TRule{\False}{\rightarrow}[2]}]
      ]
    ]
  ]
]
\end{oltableau}
\end{frame}

\begin{frame}{Applying the False \(\rightarrow\) Rule (Second Time)}
\phantomsection\label{applying-the-false-rightarrow-rule-second-time}
Now we apply the same rule to \(\mathbb{F}(Q \rightarrow P)\) on line 3.

This gives us \(\mathbb{T}Q\) and \(\mathbb{F}P\).
\end{frame}

\begin{frame}{Applying the False \(\rightarrow\) Rule (Second Time)}
\phantomsection\label{applying-the-false-rightarrow-rule-second-time-1}
\begin{oltableau}
[\sFmla{\False}{(P \rightarrow Q) \vee (Q \rightarrow P)}, 
  [\sFmla{\False}{P \rightarrow Q}, just = {\TRule{\False}{\vee}[1]}
    [\sFmla{\False}{Q \rightarrow P}, just = {\TRule{\False}{\vee}[1]}
      [\sFmla{\True}{P}, just = {\TRule{\False}{\rightarrow}[2]}
        [\sFmla{\False}{Q}, just = {\TRule{\False}{\rightarrow}[2]}
          [\sFmla{\True}{Q}, just = {\TRule{\False}{\rightarrow}[3]}
            [\sFmla{\False}{P}, just = {\TRule{\False}{\rightarrow}[3]}]
          ]
        ]
      ]
    ]
  ]
]
\end{oltableau}
\end{frame}

\begin{frame}{The Tree Closes}
\phantomsection\label{the-tree-closes}
Notice what we have now:

\begin{itemize}
\tightlist
\item
  Line 4: \(\mathbb{T}P\)
\item
  Line 7: \(\mathbb{F}P\)
\item
  Line 5: \(\mathbb{F}Q\)\\
\item
  Line 6: \(\mathbb{T}Q\)
\end{itemize}

\pause

Both \(P\) and \(Q\) have incompatible truth values. The branch closes.
(NB: We only need one inconsistency, even though we have two here.)
\end{frame}

\begin{frame}{What This Means}
\phantomsection\label{what-this-means}
\begin{itemize}[<+->]
\tightlist
\item
  We started by assuming \((P \rightarrow Q) \vee (Q \rightarrow P)\) is
  false.
\item
  Every possible way to make it false led to a contradiction.
\item
  Therefore, it \textbf{cannot} be false.
\item
  This means it's a tautology - it must always be true.
\end{itemize}
\end{frame}

\begin{frame}{Comparison to Truth Tables}
\phantomsection\label{comparison-to-truth-tables}
\begin{itemize}
\tightlist
\item
  With a truth table, we had to check 4 rows.
\item
  With the tableau, we just followed the logical implications.
\item
  For more complex sentences (with 3+ variables), tableaux can be much
  more efficient.
\item
  They don't grow exponentially like truth tables do.
\end{itemize}
\end{frame}

\begin{frame}{Example 2: Showing a Contradiction}
\phantomsection\label{example-2-showing-a-contradiction}
Let's show that \(((P \vee Q) \wedge \neg P) \wedge \neg Q\) is a
contradiction.

\begin{itemize}
\tightlist
\item
  A contradiction is a sentence that can never be true.
\item
  We'll use a tableau to demonstrate this.
\end{itemize}
\end{frame}

\begin{frame}{Strategy}
\phantomsection\label{strategy-1}
To show a sentence is a contradiction using tableaux:

\begin{itemize}[<+->]
\tightlist
\item
  Start by assuming the sentence is \textbf{true}.
\item
  If the tableau closes, then it's impossible for the sentence to be
  true.
\item
  That means the sentence must be false - it's a contradiction.
\end{itemize}
\end{frame}

\begin{frame}{Building the Tree}
\phantomsection\label{building-the-tree-1}
We start by writing that \(((P \vee Q) \wedge \neg P) \wedge \neg Q\) is
true.

\begin{oltableau}
[\sFmla{\True}{((P \vee Q) \wedge \neg P) \wedge \neg Q}]
\end{oltableau}
\end{frame}

\begin{frame}{Applying the True \(\wedge\) Rule}
\phantomsection\label{applying-the-true-wedge-rule}
Since we have a true conjunction, both conjuncts must be true.

So we get \(\mathbb{T}((P \vee Q) \wedge \neg P)\) and
\(\mathbb{T}\neg Q\).
\end{frame}

\begin{frame}{Applying the True \(\wedge\) Rule}
\phantomsection\label{applying-the-true-wedge-rule-1}
\begin{oltableau}
[\sFmla{\True}{((P \vee Q) \wedge \neg P) \wedge \neg Q}, 
  [\sFmla{\True}{(P \vee Q) \wedge \neg P}, just = {\TRule{\True}{\wedge}[1]}
    [\sFmla{\True}{\neg Q}, just = {\TRule{\True}{\wedge}[1]}]
  ]
]
\end{oltableau}
\end{frame}

\begin{frame}{Applying the True \(\wedge\) Rule Again}
\phantomsection\label{applying-the-true-wedge-rule-again}
Now we apply the rule to \(\mathbb{T}((P \vee Q) \wedge \neg P)\) from
line 2.

This gives us \(\mathbb{T}(P \vee Q)\) and \(\mathbb{T}\neg P\).
\end{frame}

\begin{frame}{Applying the True \(\wedge\) Rule Again}
\phantomsection\label{applying-the-true-wedge-rule-again-1}
\begin{oltableau}
[\sFmla{\True}{((P \vee Q) \wedge \neg P) \wedge \neg Q}, 
  [\sFmla{\True}{(P \vee Q) \wedge \neg P}, just = {\TRule{\True}{\wedge}[1]}
    [\sFmla{\True}{\neg Q}, just = {\TRule{\True}{\wedge}[1]}
      [\sFmla{\True}{P \vee Q}, just = {\TRule{\True}{\wedge}[2]}
        [\sFmla{\True}{\neg P}, just = {\TRule{\True}{\wedge}[2]}]
      ]
    ]
  ]
]
\end{oltableau}
\end{frame}

\begin{frame}{Applying the True \(\neg\) Rules}
\phantomsection\label{applying-the-true-neg-rules}
From \(\mathbb{T}\neg P\) we get \(\mathbb{F}P\), and from
\(\mathbb{T}\neg Q\) we get \(\mathbb{F}Q\).
\end{frame}

\begin{frame}
\begin{oltableau}
[\sFmla{\True}{((P \vee Q) \wedge \neg P) \wedge \neg Q}, 
  [\sFmla{\True}{(P \vee Q) \wedge \neg P}, just = {\TRule{\True}{\wedge}[1]}
    [\sFmla{\True}{\neg Q}, just = {\TRule{\True}{\wedge}[1]}
      [\sFmla{\True}{P \vee Q}, just = {\TRule{\True}{\wedge}[2]}
        [\sFmla{\True}{\neg P}, just = {\TRule{\True}{\wedge}[2]}
          [\sFmla{\False}{P}, just = {\TRule{\True}{\neg}[5]}
            [\sFmla{\False}{Q}, just = {\TRule{\True}{\neg}[3]}]
          ]
        ]
      ]
    ]
  ]
]
\end{oltableau}
\end{frame}

\begin{frame}{Applying the True \(\vee\) Rule}
\phantomsection\label{applying-the-true-vee-rule}
Now we need to apply the rule for \(\mathbb{T}(P \vee Q)\) from line 4.

This creates two branches - one where \(P\) is true, one where \(Q\) is
true.
\end{frame}

\begin{frame}
\begin{oltableau}
[\sFmla{\True}{((P \vee Q) \wedge \neg P) \wedge \neg Q}, 
  [\sFmla{\True}{(P \vee Q) \wedge \neg P}, just = {\TRule{\True}{\wedge}[1]}
    [\sFmla{\True}{\neg Q}, just = {\TRule{\True}{\wedge}[1]}
      [\sFmla{\True}{P \vee Q}, just = {\TRule{\True}{\wedge}[2]}
        [\sFmla{\True}{\neg P}, just = {\TRule{\True}{\wedge}[2]}
          [\sFmla{\False}{P}, just = {\TRule{\True}{\neg}[5]}
            [\sFmla{\False}{Q}, just = {\TRule{\True}{\neg}[3]}
              [\sFmla{\True}{P}, just = {\TRule{\True}{\vee}[4]}]
              [\sFmla{\True}{Q}, just = {\TRule{\True}{\vee}[4]}]
            ]
          ]
        ]
      ]
    ]
  ]
]
\end{oltableau}
\end{frame}

\begin{frame}{Both Branches Close}
\phantomsection\label{both-branches-close}
Look at what we have on each branch:

\textbf{Left branch:} - Line 6: \(\mathbb{F}P\) - Line 8:
\(\mathbb{T}P\)

\pause

\textbf{Right branch:} - Line 7: \(\mathbb{F}Q\) - Line 8:
\(\mathbb{T}Q\)

\pause

\textbf{Both} branches have contradictions. The entire tableau closes.
\end{frame}

\begin{frame}{What This Means}
\phantomsection\label{what-this-means-1}
\begin{itemize}[<+->]
\tightlist
\item
  We started by assuming \(((P \vee Q) \wedge \neg P) \wedge \neg Q\) is
  true
\item
  We tried every possible way to make it true (both branches)
\item
  Every possibility led to a contradiction
\item
  Therefore, the sentence \textbf{cannot} be true - it's a contradiction
\end{itemize}
\end{frame}

\begin{frame}{Understanding the Result}
\phantomsection\label{understanding-the-result}
This result makes intuitive sense:

\begin{itemize}
\tightlist
\item
  \((P \vee Q)\) says at least one of \(P\) or \(Q\) is true
\item
  \(\neg P\) says \(P\) is false
\item
  \(\neg Q\) says \(Q\) is false
\item
  You can't have at least one true when both are false.
\end{itemize}

The tableau just systematically revealed this inconsistency.
\end{frame}

\begin{frame}{Example 3: Showing a Contingent Sentence}
\phantomsection\label{example-3-showing-a-contingent-sentence}
Let's show that \(P \wedge (P \rightarrow Q)\) is neither a tautology
nor a contradiction.

\begin{itemize}
\tightlist
\item
  A contingent sentence is one that could be true or false.
\item
  We'll use two tableaux to demonstrate this.
\end{itemize}
\end{frame}

\begin{frame}{Strategy}
\phantomsection\label{strategy-2}
To show a sentence is contingent:

\begin{itemize}[<+->]
\tightlist
\item
  First, try to show it's a tautology (start with it false)
\item
  If that tableau stays open, it's not a tautology
\item
  Then, try to show it's a contradiction (start with it true)
\item
  If that tableau also stays open, it's not a contradiction
\item
  If both tableaux are open, the sentence is contingent
\end{itemize}
\end{frame}

\begin{frame}{First Tree: Is It a Tautology?}
\phantomsection\label{first-tree-is-it-a-tautology}
We start by assuming \(P \wedge (P \rightarrow Q)\) is \textbf{false}.

\begin{oltableau}
[\sFmla{\False}{P \wedge (P \rightarrow Q)}]
\end{oltableau}
\end{frame}

\begin{frame}{Applying the False \(\wedge\) Rule}
\phantomsection\label{applying-the-false-wedge-rule}
Since we have a false conjunction, at least one conjunct must be false.

This creates two branches.
\end{frame}

\begin{frame}{Applying the False \(\wedge\) Rule}
\phantomsection\label{applying-the-false-wedge-rule-1}
\begin{oltableau}
[\sFmla{\False}{P \wedge (P \rightarrow Q)}, 
  [\sFmla{\False}{P}, just = {\TRule{\False}{\wedge}[1]}]
  [\sFmla{\False}{P \rightarrow Q}, just = {\TRule{\False}{\wedge}[1]}]
]
\end{oltableau}
\end{frame}

\begin{frame}{Left Branch: Already Complete}
\phantomsection\label{left-branch-already-complete}
The left branch just has \(\mathbb{F}P\).

\begin{itemize}
\tightlist
\item
  This doesn't contradict anything.
\item
  We can't apply any more rules to it.
\item
  So this branch is \textbf{complete and open}.
\end{itemize}
\end{frame}

\begin{frame}{Right Branch: Applying False \(\rightarrow\) Rule}
\phantomsection\label{right-branch-applying-false-rightarrow-rule}
On the right branch, we have \(\mathbb{F}(P \rightarrow Q)\).

When a conditional is false, the antecedent is true and the consequent
is false.
\end{frame}

\begin{frame}{Right Branch: Applying False \(\rightarrow\) Rule}
\phantomsection\label{right-branch-applying-false-rightarrow-rule-1}
\begin{oltableau}
[\sFmla{\False}{P \wedge (P \rightarrow Q)}, 
  [\sFmla{\False}{P}, just = {\TRule{\False}{\wedge}[1]}]
  [\sFmla{\False}{P \rightarrow Q}, just = {\TRule{\False}{\wedge}[1]}
    [\sFmla{\True}{P}, just = {\TRule{\False}{\rightarrow}[2]}
      [\sFmla{\False}{Q}, just = {\TRule{\False}{\rightarrow}[2]}]
    ]
  ]
]
\end{oltableau}
\end{frame}

\begin{frame}{Right Branch: Also Complete and Open}
\phantomsection\label{right-branch-also-complete-and-open}
The right branch has \(\mathbb{F}(P \rightarrow Q)\), \(\mathbb{T}P\),
and \(\mathbb{F}Q\).

\begin{itemize}
\tightlist
\item
  These don't contradict each other.
\item
  We can't apply any more rules.
\item
  So this branch is also \textbf{complete and open}.
\end{itemize}
\end{frame}

\begin{frame}{What the First Tree Shows}
\phantomsection\label{what-the-first-tree-shows}
\begin{itemize}[<+->]
\tightlist
\item
  We assumed \(P \wedge (P \rightarrow Q)\) is false.
\item
  The tableau has open branches.
\item
  This means it \textbf{is} possible for the sentence to be false.
\item
  Therefore, it's \textbf{not a tautology}.
\end{itemize}
\end{frame}

\begin{frame}{Second Tree: Is It a Contradiction?}
\phantomsection\label{second-tree-is-it-a-contradiction}
Now we start by assuming \(P \wedge (P \rightarrow Q)\) is
\textbf{true}.

\begin{oltableau}
[\sFmla{\True}{P \wedge (P \rightarrow Q)}]
\end{oltableau}
\end{frame}

\begin{frame}{Applying the True \(\wedge\) Rule}
\phantomsection\label{applying-the-true-wedge-rule-2}
Since we have a true conjunction, both conjuncts must be true.

\begin{oltableau}
[\sFmla{\True}{P \wedge (P \rightarrow Q)}, 
  [\sFmla{\True}{P}, just = {\TRule{\True}{\wedge}[1]}
    [\sFmla{\True}{P \rightarrow Q}, just = {\TRule{\True}{\wedge}[1]}]
  ]
]
\end{oltableau}
\end{frame}

\begin{frame}{Applying the True \(\rightarrow\) Rule}
\phantomsection\label{applying-the-true-rightarrow-rule}
Now we apply the rule for \(\mathbb{T}(P \rightarrow Q)\) from line 3.

This creates two branches - one where \(P\) is false, one where \(Q\) is
true.
\end{frame}

\begin{frame}{Applying the True \(\rightarrow\) Rule}
\phantomsection\label{applying-the-true-rightarrow-rule-1}
\begin{oltableau}
[\sFmla{\True}{P \wedge (P \rightarrow Q)}, 
  [\sFmla{\True}{P}, just = {\TRule{\True}{\wedge}[1]}
    [\sFmla{\True}{P \rightarrow Q}, just = {\TRule{\True}{\wedge}[1]}
      [\sFmla{\False}{P}, just = {\TRule{\True}{\rightarrow}[3]}]
      [\sFmla{\True}{Q}, just = {\TRule{\True}{\rightarrow}[3]}]
    ]
  ]
]
\end{oltableau}
\end{frame}

\begin{frame}{Analyzing the Second Tree}
\phantomsection\label{analyzing-the-second-tree}
\begin{columns}[T]
\begin{column}{0.48\linewidth}
\textbf{Left branch:}

\begin{itemize}
\tightlist
\item
  Line 2: \(\mathbb{T}P\)
\item
  Line 4: \(\mathbb{F}P\)
\item
  This branch \textbf{closes}.
\end{itemize}
\end{column}

\begin{column}{0.48\linewidth}
\textbf{Right branch:}

\begin{itemize}
\tightlist
\item
  Line 2: \(\mathbb{T}P\)
\item
  Line 4: \(\mathbb{T}Q\)
\item
  These don't contradict anything.
\item
  This branch is \textbf{complete and open}.
\end{itemize}
\end{column}
\end{columns}
\end{frame}

\begin{frame}{What the Second Tree Shows}
\phantomsection\label{what-the-second-tree-shows}
\begin{itemize}[<+->]
\tightlist
\item
  We assumed \(P \wedge (P \rightarrow Q)\) is true.
\item
  The tableau has an open branch.
\item
  This means it \textbf{is} possible for the sentence to be true.
\item
  Therefore, it's \textbf{not a contradiction}.
\end{itemize}
\end{frame}

\begin{frame}{Conclusion}
\phantomsection\label{conclusion}
Since \(P \wedge (P \rightarrow Q)\):

\begin{itemize}[<+->]
\tightlist
\item
  Is not a tautology (first tree has open branches),
\item
  Is not a contradiction (second tree has an open branch),
\item
  It must be \textbf{contingent} - sometimes true, sometimes false.
\end{itemize}

\pause

The open branches actually show us \textbf{when} it's true or false:

\begin{itemize}
\tightlist
\item
  It's false when \(P\) is false (first tree, left branch)
\item
  It's true when both \(P\) and \(Q\) are true (second tree, right
  branch).
\end{itemize}
\end{frame}

\begin{frame}{Comparison to Truth Tables}
\phantomsection\label{comparison-to-truth-tables-1}
\begin{itemize}[<+->]
\tightlist
\item
  A \textbf{tautology} has a T under the main connective in every row.
\item
  A \textbf{contradiction} has an F under the main connective in every
  row.
\item
  A \textbf{contingent} sentence has a mixture Ts and Fs under the main
  connective.
\end{itemize}
\end{frame}

\begin{frame}{For Next Time}
\phantomsection\label{for-next-time}
We'll use trees to test for validity.

Remember there is no lecture on Wednesday, but there is a discussion
section.
\end{frame}




\end{document}
