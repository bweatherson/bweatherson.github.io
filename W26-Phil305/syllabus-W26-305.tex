% Options for packages loaded elsewhere
% Options for packages loaded elsewhere
\PassOptionsToPackage{unicode}{hyperref}
\PassOptionsToPackage{hyphens}{url}
\PassOptionsToPackage{dvipsnames,svgnames,x11names}{xcolor}
%
\documentclass[
  letterpaper,
  DIV=11,
  numbers=noendperiod]{scrartcl}
\usepackage{xcolor}
\usepackage[margin=1in]{geometry}
\usepackage{amsmath,amssymb}
\setcounter{secnumdepth}{-\maxdimen} % remove section numbering
\usepackage{iftex}
\ifPDFTeX
  \usepackage[T1]{fontenc}
  \usepackage[utf8]{inputenc}
  \usepackage{textcomp} % provide euro and other symbols
\else % if luatex or xetex
  \usepackage{unicode-math} % this also loads fontspec
  \defaultfontfeatures{Scale=MatchLowercase}
  \defaultfontfeatures[\rmfamily]{Ligatures=TeX,Scale=1}
\fi
\usepackage{lmodern}
\ifPDFTeX\else
  % xetex/luatex font selection
  \setmainfont[]{Atkinson Hyperlegible}
  \setmathfont[]{STIX Two Math}
\fi
% Use upquote if available, for straight quotes in verbatim environments
\IfFileExists{upquote.sty}{\usepackage{upquote}}{}
\IfFileExists{microtype.sty}{% use microtype if available
  \usepackage[]{microtype}
  \UseMicrotypeSet[protrusion]{basicmath} % disable protrusion for tt fonts
}{}
\makeatletter
\@ifundefined{KOMAClassName}{% if non-KOMA class
  \IfFileExists{parskip.sty}{%
    \usepackage{parskip}
  }{% else
    \setlength{\parindent}{0pt}
    \setlength{\parskip}{6pt plus 2pt minus 1pt}}
}{% if KOMA class
  \KOMAoptions{parskip=half}}
\makeatother
% Make \paragraph and \subparagraph free-standing
\makeatletter
\ifx\paragraph\undefined\else
  \let\oldparagraph\paragraph
  \renewcommand{\paragraph}{
    \@ifstar
      \xxxParagraphStar
      \xxxParagraphNoStar
  }
  \newcommand{\xxxParagraphStar}[1]{\oldparagraph*{#1}\mbox{}}
  \newcommand{\xxxParagraphNoStar}[1]{\oldparagraph{#1}\mbox{}}
\fi
\ifx\subparagraph\undefined\else
  \let\oldsubparagraph\subparagraph
  \renewcommand{\subparagraph}{
    \@ifstar
      \xxxSubParagraphStar
      \xxxSubParagraphNoStar
  }
  \newcommand{\xxxSubParagraphStar}[1]{\oldsubparagraph*{#1}\mbox{}}
  \newcommand{\xxxSubParagraphNoStar}[1]{\oldsubparagraph{#1}\mbox{}}
\fi
\makeatother


\usepackage{longtable,booktabs,array}
\usepackage{calc} % for calculating minipage widths
% Correct order of tables after \paragraph or \subparagraph
\usepackage{etoolbox}
\makeatletter
\patchcmd\longtable{\par}{\if@noskipsec\mbox{}\fi\par}{}{}
\makeatother
% Allow footnotes in longtable head/foot
\IfFileExists{footnotehyper.sty}{\usepackage{footnotehyper}}{\usepackage{footnote}}
\makesavenoteenv{longtable}
\usepackage{graphicx}
\makeatletter
\newsavebox\pandoc@box
\newcommand*\pandocbounded[1]{% scales image to fit in text height/width
  \sbox\pandoc@box{#1}%
  \Gscale@div\@tempa{\textheight}{\dimexpr\ht\pandoc@box+\dp\pandoc@box\relax}%
  \Gscale@div\@tempb{\linewidth}{\wd\pandoc@box}%
  \ifdim\@tempb\p@<\@tempa\p@\let\@tempa\@tempb\fi% select the smaller of both
  \ifdim\@tempa\p@<\p@\scalebox{\@tempa}{\usebox\pandoc@box}%
  \else\usebox{\pandoc@box}%
  \fi%
}
% Set default figure placement to htbp
\def\fps@figure{htbp}
\makeatother





\setlength{\emergencystretch}{3em} % prevent overfull lines

\providecommand{\tightlist}{%
  \setlength{\itemsep}{0pt}\setlength{\parskip}{0pt}}



 


\linespread{1.05}
\RedeclareSectionCommand[
  afterskip=0.3ex,
  beforeskip=0.8ex
  ]{subsection}
\RedeclareSectionCommand[
  afterskip=0.1ex,
  beforeskip=0.5ex
  ]{subsubsection}
\KOMAoption{captions}{tableheading}
\makeatletter
\@ifpackageloaded{caption}{}{\usepackage{caption}}
\AtBeginDocument{%
\ifdefined\contentsname
  \renewcommand*\contentsname{Table of contents}
\else
  \newcommand\contentsname{Table of contents}
\fi
\ifdefined\listfigurename
  \renewcommand*\listfigurename{List of Figures}
\else
  \newcommand\listfigurename{List of Figures}
\fi
\ifdefined\listtablename
  \renewcommand*\listtablename{List of Tables}
\else
  \newcommand\listtablename{List of Tables}
\fi
\ifdefined\figurename
  \renewcommand*\figurename{Figure}
\else
  \newcommand\figurename{Figure}
\fi
\ifdefined\tablename
  \renewcommand*\tablename{Table}
\else
  \newcommand\tablename{Table}
\fi
}
\@ifpackageloaded{float}{}{\usepackage{float}}
\floatstyle{ruled}
\@ifundefined{c@chapter}{\newfloat{codelisting}{h}{lop}}{\newfloat{codelisting}{h}{lop}[chapter]}
\floatname{codelisting}{Listing}
\newcommand*\listoflistings{\listof{codelisting}{List of Listings}}
\makeatother
\makeatletter
\makeatother
\makeatletter
\@ifpackageloaded{caption}{}{\usepackage{caption}}
\@ifpackageloaded{subcaption}{}{\usepackage{subcaption}}
\makeatother
\usepackage{bookmark}
\IfFileExists{xurl.sty}{\usepackage{xurl}}{} % add URL line breaks if available
\urlstyle{same}
\hypersetup{
  pdftitle={PHIL 305: Introduction to Formal Methods},
  pdfauthor={Brian Weatherson},
  colorlinks=true,
  linkcolor={blue},
  filecolor={Maroon},
  citecolor={Blue},
  urlcolor={Blue},
  pdfcreator={LaTeX via pandoc}}


\title{PHIL 305: Introduction to Formal Methods}
\author{Brian Weatherson}
\date{Winter 2026}
\begin{document}
\maketitle


\textbf{Lead Instructor}: Brian Weatherson\\
\textbf{Email}: weath@umich.edu\\
\textbf{Web}: canvas.umich.edu\\
\textbf{Office Hours}: Thursday 9.30-11.30 (Angell Hall, Rm 2207)\\
\textbf{Discussion Section Leader}: Brett Thompson\\
\textbf{Email}: brettdt@umich.edu\\
\textbf{Office Hours}: Monday 4-6

~

\section{Key Points}\label{key-points}

\begin{itemize}
\tightlist
\item
  The course has four parts:

  \begin{enumerate}
  \def\labelenumi{\arabic{enumi}.}
  \tightlist
  \item
    Propositional Logic;
  \item
    Modal Logic;
  \item
    Probability and Decision; and,
  \item
    Game Theory.
  \end{enumerate}
\item
  Each part of the course has a different textbook.
\item
  All the books are open access; they are freely available and linked
  through Canvas.
\item
  There are 12 weekly assignments, due each week on Thursday at 5pm. The
  best 10 count for your final grade.
\item
  There is a midterm test, in the last lecture before the break, and
  which counts for 10\%.
\item
  There is a final exam, which counts for 20\%, and which covers all the
  material in the course.
\end{itemize}

\section{Course Description}\label{course-description}

This course will introduce some important formal tools that are used
elsewhere in philosophy. We will look at propositional logic, the logic
of modals and conditionals, probability theory and game theory.
Obviously that's a lot to cover in a short time. The aim here is to make
sure you understand the basics, and the symbolism, so you can follow
simple applications of these tools, and you have the foundations to
understand more complicated applications.

\section{Required Materials}\label{required-materials}

There are four textbooks for the course. All of them are open access,
and hence free. The second is on Canvas, the other three have to be
downloaded from elsewhere.

\begin{itemize}
\tightlist
\item
  \emph{forall x: Calgary} by P. D. Magnus, Tim Button, J. Robert
  Loftis, Robert Trueman, Aaron Thomas-Bolduc and Richard Zach.
  Available at \url{http://forallx.openlogicproject.org}.
\item
  \emph{Boxes and Diamonds: Ann Arbor remix} by Richard Zach and edited
  by Brian Weatherson. Available on Canvas.
\item
  \emph{Odds and Ends} by Jonathan Weisberg. Available at
  \url{https://jonathanweisberg.org/vip/}.
\item
  \emph{Game Theory} by Giacomo Bonnano. Available at
  \url{https://faculty.econ.ucdavis.edu/faculty/bonanno/GT_Book.html}.
\end{itemize}

\section{Course Requirements}\label{course-requirements}

\begin{itemize}
\tightlist
\item
  There will be 12 weekly assignments. Of these your best 10 will count
  towards your grade, with each counting for about 7\%.
\item
  There will be a midterm exam which counts for 10\% of the grade.
\item
  There will also be a final exam that counts for 20\% of the grade.
\end{itemize}

\section{Summary of Grading System}\label{summary-of-grading-system}

\begin{enumerate}
\def\labelenumi{\arabic{enumi}.}
\tightlist
\item
  Weekly assignments - 7\% each, 10 assignments count, 70\% total.
\item
  Midterm exam - 10\%.
\item
  Final exam - 20\%.
\end{enumerate}

\section{Canvas}\label{canvas}

There is a Canvas site for this course, which can be accessed from
\url{https://canvas.umich.edu}. Course documents (syllabus, lecture
notes, assignments) will be available from this site. Please make sure
that you can access this site. Consult the site regularly for
announcements, including changes to the course schedule. And there are
many tools on the site to communicate with each other, and with me.

\pagebreak

\section{Class Schedule}\label{class-schedule}

The readings will all be from the four textbooks. You should do the
reading before the scheduled class.

\subsection{Week 1: Introduction}\label{week-1-introduction}

\subsubsection{Wednesday, January 07}\label{wednesday-january-07}

\textbf{Topic}: Introduction\\
\textbf{Reading}: \emph{forall x}, Chapters 1-2.

\subsection{Week 2: Symbols and Truth
Tables}\label{week-2-symbols-and-truth-tables}

\subsubsection{Monday, January 12}\label{monday-january-12}

\textbf{Topic}: Symbolization\\
\textbf{Reading}: \emph{forall x}, Chapters 4-6.

\subsubsection{Wednesday, January 14}\label{wednesday-january-14}

\textbf{Topic}: Truth Tables\\
\textbf{Reading}: \emph{forall x}, Chapters 9-11.\\
\textbf{Assignment}: Assignment 1 due Thursday, January 15 at 5pm.

\subsection{Week 3: Validity}\label{week-3-validity}

\subsubsection{Monday, January 19}\label{monday-january-19}

No class for MLK Day

\subsubsection{Wednesday, January 21}\label{wednesday-january-21}

\textbf{Topic}: Truth Tables and Validity\\
\textbf{Reading}: \emph{forall x}, Chapter 12.\\
\textbf{Assignment}: Assignment 2 due Thursday, January 22 at 5pm.

\subsection{Week 4: Trees}\label{week-4-trees}

\subsubsection{Monday, January 26}\label{monday-january-26}

\textbf{Topic}: Truth Trees\\
\textbf{Reading}: \emph{Boxes and Diamonds}, Sections 2.1-2.3

\subsubsection{Wednesday, January 28}\label{wednesday-january-28}

No lecture, because I'm away at a conference. There will be a discussion
section.\\
\textbf{Assignment}: Assignment 3 due Thursday, January 29 at 5pm.

\subsection{Week 5: Trees and Modality}\label{week-5-trees-and-modality}

\subsubsection{Monday, February 02}\label{monday-february-02}

\textbf{Topic}: Using Truth Trees\\
\textbf{Reading}: \emph{Boxes and Diamonds}, Sections 2.4-2.5

\subsubsection{Wednesday, February 04}\label{wednesday-february-04}

\textbf{Topic}: Varieties of Modality\\
\textbf{Reading}: \emph{Boxes and Diamonds}, sections 3.1-3.3.\\
\textbf{Assignment}: Assignment 4 (on trees) due Thursday, January 29 at
5pm.

\subsection{Week 6: Validity in Modal
Logic}\label{week-6-validity-in-modal-logic}

\subsubsection{Monday, February 09}\label{monday-february-09}

\textbf{Topic}: Models and Frames\\
\textbf{Reading}: \emph{Boxes and Diamonds}, sections 3.4-4.5.

\subsubsection{Wednesday, February 11}\label{wednesday-february-11}

\textbf{Topic}: Modal Tableau\\
\textbf{Reading}: \emph{Boxes and Diamonds}, chapter 5.\\
\textbf{Assignment}: Assignment 5 due Thursday, February 12 at 5pm.

\subsection{Week 7: Conditionals}\label{week-7-conditionals}

\subsubsection{Monday, February 16}\label{monday-february-16}

\textbf{Topic}: Examples of Modal Tableau\\
\textbf{Reading}: No new reading

\subsubsection{Wednesday, February 18}\label{wednesday-february-18}

\textbf{Topic}: Varieties of Conditionals\\
\textbf{Reading}: \emph{Boxes and Diamonds}, chapter 6.\\
\textbf{Assignment}: Assignment 6 (on modal tableau) due Thursday,
February 19 at 5pm.

\subsection{Week 8: Counterfactuals}\label{week-8-counterfactuals}

\subsubsection{Monday, February 16}\label{monday-february-16-1}

\textbf{Topic}: Counterfactual Conditionals\\
\textbf{Reading}: \emph{Boxes and Diamonds}, chapter 7.

\subsubsection{Wednesday, February 25}\label{wednesday-february-25}

\textbf{MIDTERM EXAM}

\textbf{Assignment}: Assignment 7 due Thursday, February 26 at 5pm.

\subsection{Week 9: Starting on
Probability}\label{week-9-starting-on-probability}

\subsubsection{Monday, March 09}\label{monday-march-09}

\textbf{Topic}: Probability Basics\\
\textbf{Reading}: \emph{Odds and Ends}, chapters 1 and 5

\subsubsection{Wednesday, March 11}\label{wednesday-march-11}

\textbf{Topic}: Probability Trees\\
\textbf{Reading}: \emph{Odds and Ends}, chapter 1\\
\textbf{Assignment}: No assignment this week.

\subsection{Week 10: Conditional
Probability}\label{week-10-conditional-probability}

\subsubsection{Monday, March 16}\label{monday-march-16}

\textbf{Topic}: Conditional Probability\\
\textbf{Reading}: \emph{Odds and Ends}, chapter 6

\subsubsection{Wednesday, March 18}\label{wednesday-march-18}

\textbf{Topic}: Inverting Conditional Probability\\
\textbf{Reading}: \emph{Odds and Ends}, chapter 8\\
\textbf{Assignment}: Assignment 8 due Thursday, March 19 at 5pm.

\subsection{Week 11: Probability and
Decision}\label{week-11-probability-and-decision}

\subsubsection{Monday, March 23}\label{monday-march-23}

\textbf{Topic}: Expected Utility\\
\textbf{Reading}: \emph{Odds and Ends}, chapters 11 and 12

\subsubsection{Wednesday, March 25}\label{wednesday-march-25}

\textbf{Topic}: Utility and Money\\
\textbf{Reading}: \emph{Odds and Ends}, sections 12.5 and 13.1.\\
\textbf{Assignment}: Assignment 9 due Thursday, March 26 at 5pm.

\pagebreak

\subsection{Week 12: What are Games?}\label{week-12-what-are-games}

\subsubsection{Monday, March 30}\label{monday-march-30}

\textbf{Topic}: Games, Payouts and Utilities\\
\textbf{Reading}: \emph{Game Theory}, sections 2.1 and 2.2.

\subsubsection{Wednesday, April 01}\label{wednesday-april-01}

\textbf{Topic}: Dominant Strategies and Equilibria\\
\textbf{Reading}: \emph{Game Theory}, sections 2.5 and 2.6.\\
\textbf{Assignment}: Assignment 10 due Thursday, April 02 at 5pm.

\subsection{Week 13: Games and Time}\label{week-13-games-and-time}

\subsubsection{Monday, April 06}\label{monday-april-06}

\textbf{Topic}: Dynamic Games\\
\textbf{Reading}: \emph{Game Theory}, sections 3.1-3.3.

\subsubsection{Wednesday, April 08}\label{wednesday-april-08}

\textbf{Topic}: Backward Induction\\
\textbf{Reading}: \emph{Game Theory}, sections 3.4 and 3.5.\\
\textbf{Assignment}: Assignment 11 due Thursday, April 09 at 5pm.

\subsection{Week 14: Games with Cardinal
Payoffs}\label{week-14-games-with-cardinal-payoffs}

\subsubsection{Monday, April 13}\label{monday-april-13}

\textbf{Topic}: Nash Equilibria\\
\textbf{Reading}: \emph{Game Theory}, sections 6.1‑6.3.

\subsubsection{Wednesday, April 15}\label{wednesday-april-15}

\textbf{Topic}: Signaling Games\\
\textbf{Reading}: No new reading\\
\textbf{Assignment}: Assignment 12 due Thursday, April 16 at 5pm.

There will be a revision class on Monday, April 20, with no new
material, to go over the sample exam.

\section{Extensions}\label{extensions}

Because we will release the answers to the quizzes at the due date, we
won't be allowing extensions on them.

That's why we're only counting the best 10 out of 12 assignments. People
who are sick, or have other obligations, can skip up to two assignments
without penalty.

\section{Plagiarism}\label{plagiarism}

Although team-work, and even co-authorship, is encouraged, plagiarism is
strictly prohibited. You are responsible for making sure that none of
your work is plagiarized. Be sure to cite work that you use, both direct
quotations and paraphrased ideas. Any citation method that is tolerably
clear is permitted, but if you'd like a good description of a citation
scheme that works well in philosophy, look at
\url{http://bit.ly/VDhRJ4}.

You are encouraged to discuss the course material, including
assignments, with your classmates, but all written work that you hand in
under your own name must be your own. If work is handed is as the work
of two people, you are affirming that each person did a fair share of
the work. (Note that when you're submitting work on Canvas, you have to
each submit the paper, even if it is co-authored. That way Canvas knows
that everyone has turned in work.)

You should also be familiar with the academic integrity policies of the
College of Literature, Science \& the Arts at the University of
Michigan, which are available here:
\url{http://www.lsa.umich.edu/academicintegrity/}. Violations of these
policies will be reported to the Office of the Assistant Dean for
Student Academic Affairs, and sanctioned with a course grade of F.

\section{Disability}\label{disability}

The University of Michigan abides by the Americans with Disabilities Act
of 1990, Section 504 of the Rehabilitation Act of 1973, and other
applicable federal and state laws that prohibit discrimination on the
basis of disability, which mandate that reasonable accommodations be
provided for qualified students with disabilities.

If you have a disability, and may require some type of instructional
and/or examination accommodation, please contact me early in the
semester. If you have not already done so, you will also need to
register with the Office of Services for Students with Disabilities. The
office is located at G664 Haven Hall.

For more information on disability services at the University of
Michigan, go to \url{http://ssd.umich.edu}.




\end{document}
